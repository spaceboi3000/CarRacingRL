\section{Αποτελέσματα}

\subsection{Standard Ρύθμιση}

\begin{itemize}
    \item \textbf{Device:} cpu
    \item \textbf{Loaded model from:} 'dreamer\_best.pth'
\end{itemize}

Η εκπαίδευση πραγματοποιήθηκε με την ακόλουθη διαμόρφωση:

\begin{table}[H]
    \centering
    % Ρύθμιση μικρότερης γραμματοσειράς
    \footnotesize 
    % Μείωση του ύψους των γραμμών για πιο συμπαγές αποτέλεσμα
    \renewcommand{\arraystretch}{1.1} 
    
    % Χρήση tabularx για να ορίσουμε ακριβώς το 70% του πλάτους κειμένου
    % Ο τύπος 'X' προσαρμόζει αυτόματα το πλάτος της δεύτερης στήλης
    \begin{tabularx}{0.7\textwidth}{|l|X|}
        \hline
        \textbf{Παράμετρος} & \textbf{Τιμή} \\
        \hline
        \multicolumn{2}{|c|}{\cellcolor{lightGray}\textit{Περιβάλλον}} \\
        \hline
        Environment & CarRacing-V3 \\
        Observation & $64 \times 64$ grayscale \\
        Action space & [steering, gas, brake] $\in [-1,1]^3$ \\
        Action repeat & 2 \\
        \hline
        \multicolumn{2}{|c|}{\cellcolor{lightGray}\textit{Αρχιτεκτονική}} \\
        \hline
        Stochastic dim (z) & 30 \\
        Deterministic dim (h) & 200 \\
        Hidden dim & 300 \\
        Embedding dim & 1024 \\
        \hline
        \multicolumn{2}{|c|}{\cellcolor{lightGray}\textit{Εκπαίδευση}} \\
        \hline
        Total steps & 500,000 (target) / 23,500 (actual) \\
        Seed episodes & 5 \\
        Batch size & 50 \\
        Sequence length & 50 \\
        Imagination horizon & 15 \\
        \hline
        \multicolumn{2}{|c|}{\cellcolor{lightGray}\textit{Hyperparameters}} \\
        \hline
        World model LR & $6 \times 10^{-4}$ \\
        Actor LR & $8 \times 10^{-5}$ \\
        Critic LR & $8 \times 10^{-5}$ \\
        Discount ($\gamma$) & 0.99 \\
        Lambda ($\lambda$) & 0.95 \\
        Free nats & 3.0 \\
        Gradient clip & 100 \\
        \hline
        \multicolumn{2}{|c|}{\cellcolor{lightGray}\textit{Hardware}} \\
        \hline
        Device & CPU \\
        Training time & $\sim$ 48 hours \\
        \hline
    \end{tabularx}
    \caption{Παράμετροι εκπαίδευσης}
\end{table}

\subsection{Σύνοψη Εκπαίδευσης}
Το μονέλο εκπαιδεύτηκε συνολικά για 47 επεισόδια, με βήματα που κυμαίνονται από 500 έως 23,500. 
Τα στατιστικά εκπαίδευσης δείχνουν σημαντική μείωση στο σφάλμα ανακατασκευής του μοντέλου, αν και το σήμα επιβράβευσης παρουσιάζει υψηλή διακύμανση, 
κάτι που είναι τυπικό για την αρχική φάση της ενισχυτικής μάθησης σε σύνθετα περιβάλλοντα όπως το \texttt{CarRacing-V3}.   


\subsection{Σύγκριση Επιδόσεων (Performance Comparison)}

\begin{table}[H]
    \centering
    \begin{tabular}{@{}lcc@{}}
        \toprule
        \textbf{Metric} & \textbf{Initial Model} & \textbf{Optimized Model (Final)} \\
        \midrule
        Mean Reward & -42.07 & \textbf{670.21} \\
        Std Deviation & 8.17 & 165.03 \\
        Min Reward & -53.18 & 372.13 \\
        Max Reward & -27.97 & 837.04 \\
        \bottomrule
    \end{tabular}
    \caption{Σύγκριση επιδόσεων μεταξύ αρχικού και τελικού μοντέλου.}
    \label{tab:comparison}
\end{table}

Αναλυτικά τα 5 επεισόδια της τελικής αξιολόγησης:

\begin{table}[H]
    \centering
    \begin{tabular}{ccr}
        \toprule
        \textbf{Episode} & \textbf{Steps} & \textbf{Reward} \\
        \midrule
        1 & 1000 & 837.04 \\
        2 & 1000 & 654.97 \\
        3 & 1000 & 678.17 \\
        4 & 1000 & 372.13 \\
        5 & 1000 & 808.75 \\
        \bottomrule
    \end{tabular}
    \caption{Αναλυτικά αποτελέσματα του βελτιστοποιημένου μοντέλου.}
\end{table}

Το υψηλό μέσο reward (670.21) υποδεικνύει ότι ο πράκτορας έμαθε επιτυχώς 
να πλοηγείται στην πίστα, παραμένοντας εντός δρόμου και διατηρώντας ικανοποιητική ταχύτητα.

\subsection{Οπτική Παρατήρηση}
Παρακάτω δείχνουμε κάποιες εικόνες από το περιβάλλον προσομοίωσης με μία γρήγορη εκπαίδευση 5000 βημάτων. Η πρώτη μας δείχνει ότι το αυτοκίνητο παρόλο που ο δρόμος είναι ευθεία πάει να στρίψει. 
Στην δεύτερη έχει βγει εκτός πορείας.

\begin{figure}[h]
    \centering
    \begin{subfigure}[b]{0.4\textwidth}
        \centering
        \includegraphics[width=\textwidth]{figures/run1.png}
        \caption{Αυτοκίνητο που στρίβει σε ευθεία}
    \end{subfigure}
    \hfill
    \begin{subfigure}[b]{0.3\textwidth}
        \centering
        \includegraphics[width=\textwidth]{figures/run2.png}
        \caption{Αυτοκίνητο εκτός πορείας}
    \end{subfigure}
    \caption{Εικόνες από την προσομοίωση}
\end{figure}
Παρατηρούμε λοιπόν ότι το μοντέλο δεν έχει μάθει ακόμα να κρατιέται στην πορεία, δεν καταλαβαίνει καν τι χρειάζεται να κάνει.
Αυτό είναι λογικό και αναμενόμενο αφού η μέθοδος του Dreamer απαιτεί πολλές ώρες εκπαίδευσης και μεγάλο αριθμό βημάτων δοκιμής
και αποτυχίας μέχρι να φτάσει σε ικανοποιητικά αποτελέσματα.

Παρακάτω βλέπουμε πώς απεικονίζονται τα όνειρα κατά την διάρκεια του testing.
\begin{figure}[H]
    \centering
    % Εικόνα 1
    \begin{subfigure}[b]{0.8\textwidth}
        \centering
        \includegraphics[width=\textwidth]{figures/Dreams1.png}
        \caption{Αυτοκίνητο που κινείται κανονικά σε ευθεία, 5 όνειρα και ανακατασκευή}
    \end{subfigure}
    
    \vspace{1cm} 
    
    % Εικόνα 2
    \begin{subfigure}[b]{0.8\textwidth}
        \centering
        \includegraphics[width=\textwidth]{figures/Dreams2.png}
        \caption{Αυτοκίνητο οριακά εκτός πορείας σε ευθεία, 5 όνειρα και ανακατασκευή}
    \end{subfigure}
    \pagebreak
    \vspace{1cm}
    \caption{Εικόνες από το testing με όνειρα και ανακατασκευές}
\end{figure}

Στη συνέχεια πραγματοποιήσαμε εκπαίδευση για 23500 βήματα 
και παρακάτω παρουσιάζουμε τα όνειρα (dreams) και τις ανακατασκευές (reconstructions) που παράγει το μοντέλο.

\begin{figure}[H]
    \centering
    % Εικόνα 3
    \begin{subfigure}[b]{0.8\textwidth}
        \centering
        \includegraphics[width=\textwidth]{figures/Dreamsonly1.png}
        \caption{Αυτοκίνητο εντός πορείας σε στροφή και 5 όνειρα}
    \end{subfigure}

    \vspace{1cm}

    % Εικόνα 4
    \begin{subfigure}[b]{0.8\textwidth}
        \centering
        \includegraphics[width=\textwidth]{figures/Dreamsonly2.png}
        \caption{Αυτοκίνητο οριακά εντός πορείας σε ευθεία και 5 όνειρα}
    \end{subfigure}

    \vspace{1cm}
    \caption{Εικόνες από το testing με όνειρα}
\end{figure}

Παρατηρούμε ότι με 23000 βήματα εκπαίδευσης τα αποτελέσματα είναι πολύ καλύτερα αλλά κάποιες φορές πάει να βγει εκτός πορείας.
Τέλος, παρακάτω παρουσιάζουμε όνειρα και ανακατασκευές από ένα καλύτερο run.
\begin{figure}[H]
    \centering
    \centering
    \includegraphics[width=\textwidth]{figures/Dreams5.png} 
    \caption{Επιτυχημένη οδήγηση με όνειρα και ανακατασκευή}
\end{figure}

\begin{figure}[H]
    \centering
    \centering
    \includegraphics[width=\textwidth]{figures/Dreamsfull1.png} 
    \caption{Επιτυχημένη οδήγηση με όνειρα και ανακατασκευή (απότομη στροφή)} 
\end{figure}
Παρατηρούμε ότι τα όνειρα διατηρούν βασικά χαρακτηριστικά της πραγματικής εικόνας, όπως η θέση του δρόμου και του αυτοκινήτου,
αλλά ταυτόχρονα παρουσιάζουν παραμορφώσεις και απώλειες λεπτομερειών, όσο εξελισσεται ο δρόμος και ειδικά προς τις στροφές.
Αυτή είναι η δουλειά του ονείρου να προβλέπει την μελλοντική κατάσταση βασισμένο στην τρέχουσα, χωρίς να έχει απόλυτη βεβαιότητα/ακριβεια.

Οι ανακατασκευές είναι αρκετά πιστές στις αρχικές εικόνες, υποδεικνύοντας ότι το VAE κατάφερε να μάθει μια αποτελεσματική 
αναπαράσταση του περιβάλλοντος τόσο για κωδικοποίηση όσο και για ανακατασκευή.

Ακόμα και σε κλειστές στροφές βρίσκει τον τρόπο να παραμείνει στην πορεία, δείχνοντας ότι ο πράκτορας έχει μάθει μια 
ικανοποιητική πολιτική οδήγησης.