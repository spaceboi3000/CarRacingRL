\section{Συμπεράσματα}

Στην παρούσα εργασία υλοποιήσαμε τον αλγόριθμο Google Dreamer V1 από την αρχή και τον εφαρμόσαμε στο περιβάλλον \texttt{CarRacing-V3} του Gymnasium. Μέσω της ανάλυσης της εκπαιδευτικής διαδικασίας και των αποτελεσμάτων, καταλήγουμε στα ακόλουθα συμπεράσματα.

\subsection{Επιτεύγματα}
Η υλοποίηση ανέδειξε τη δύναμη των μοντέλων λανθάνοντος χώρου:
\begin{itemize}
    \item \textbf{Εκμάθηση Δυναμικής:} Το world model κατάφερε επιτυχώς να μάθει τη δυναμική του περιβάλλοντος, 
    γεγονός που επιβεβαιώνεται από τη σημαντική μείωση του reconstruction loss.
    \item \textbf{Latent Imagination:} Η αρχιτεκτονική RSSM επέτρεψε την εκπαίδευση του πράκτορα «στη φαντασία» (λανθάνοντα χώρο), 
    μειώνοντας δραστικά την ανάγκη για συνεχή αλληλεπίδραση με το πραγματικό περιβάλλον.
    \item \textbf{Αποδοτικότητα Πόρων:} Η στρατηγική χρήση ασπρόμαυρων εικόνων (grayscale) 
    κατέστησε εφικτή την εκπαίδευση του μοντέλου σε λογικό χρόνο, ακόμα και χωρίς τη χρήση GPU.
\end{itemize}

\begin{figure}[H]
    \centering
    \includegraphics[width=0.9\linewidth]{figures/training_overview.png}
    \caption{Συνολική επισκόπηση της πορείας εκπαίδευσης (Training Overview).}
    \label{fig:conclusion_overview}
\end{figure}