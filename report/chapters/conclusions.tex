\section{Συμπεράσματα}

Στην παρούσα εργασία υλοποιήσαμε τον αλγόριθμο Google Dreamer V1 από την αρχή και τον εφαρμόσαμε 
στο περιβάλλον \texttt{CarRacing-V3} του Gymnasium.
Η απόδοση του Google Dreamer V1 ανέδειξε τη σημαντική υπεροχή των Model-Based προσεγγίσεων σε 
περιβάλλοντα περιορισμένων πόρων. 

Η χρήση του Λανθάνοντος Χώρου (Latent Space) επέτρεψε στον πράκτορα να γενικεύει αποτελεσματικά, 
μαθαίνοντας να οδηγεί ακόμη και όταν οι προβλέψεις του (όνειρα) γίνονταν ασαφείς σε βάθος χρόνου. 
Η δυνατότητα αυτή επιταχύνει την εκπαίδευση και μειώνει την ανάγκη για εκτεταμένη αλληλεπίδραση με 
το περιβάλλον.
Η στρατηγική επιλογή της ασπρόμαυρης επεξεργασίας απέδειξε ότι η γεωμετρική πληροφορία επαρκεί 
για την επίλυση του CarRacing-V3, μειώνοντας δραστικά το υπολογιστικό κόστος. 
Τέλος, διαπιστώθηκε πρακτικά ότι η σωστή διάδοση των gradients μέσω των φανταστικών τροχιών 
(backpropagation through time) είναι απαραίτητη για την επιτυχή εκπαίδευση του Actor, καθώς χωρίς αυτήν ο πράκτορας αδυνατεί 
να συνδέσει τις ενέργειές του με τα μελλοντικά αποτελέσματα.

Η αλληλεξάρτηση των διαφόρων συνιστωσών του Dreamer (VAE, RSSM, Actor, Critic) κατά την εκπαίδευση
είναι συγχρόνως προτέρημα και πρόκληση, γεγονός που κάνει τον Google Dreamer V1 έναν ισχυρό και
έξυπνο αλγόριθμο για την επίλυση σύνθετων προβλημάτων ενισχυτικής μάθησης.

Θα ήταν ενδιαφέρον να τον εκπαιδεύσουμε και σε άλλα περιβάλλοντα όπως το "DOOM" ή το "Atari"
ωστόσο λόγω των περιορισμένων υπολογιστικών πόρων δεν ήταν εφικτό να το πραγματοποιήσουμε στο πλαίσιο αυτής της εργασίας.   